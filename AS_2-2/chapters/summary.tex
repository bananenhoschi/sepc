\chapter*{Zusammenfassung}
\thispagestyle{fancy}

Diese Arbeit liefert eine Gegenüberstellung von Scrum und Disciplined Agile Delivery. Der Fokus wird auf die Themen Planung, Zusammenarbeit und Zuständigkeit gelegt. Es werden Grundkenntnisse über das Vorgehensmodell Scrum oder Extrem Programming (XP) vorausgesetzt.

Disciplined Agile Delivery (DAD) basiert grösstenteils auf Scrum und Extreme Programming. Im Zentrum steht die Construction Phase mit Iterationen (Sprints) von fixer Zeitdauer. Dieses Vorgehensmodel findet dann Anwendung, wenn Scrum den Bedürfnissen nicht mehr genügt und erweitert werden muss und ein agiles Projekt durchgeführt werden muss.

DAD ist \textbf{iterationsbasiert}. Wie bei vielen agilen Methoden, einschliesslich Scrum und XP, wird die Lösung schrittweise und zeitgesteuert aufgebaut. Diese Zeitrahmen werden Iterationen genannt (was Scrum Sprints nennt).

DAD verwendet \textbf{nicht die Scrum Terminologie}. Die Entwickler von DAD haben sich explizit gegen die Scrum Terminologie entschieden. Jedoch spielt diese keine Rolle. So darf in DAD mit der Scrum Terminologie gearbeitet werden. Im Blog «Disciplined Agile Terminology»\cite{blogTerminology} wird erklärt warum das Team um Scott Ambler sich gegen die Scrum Terminologie entschieden hat.

DAD zeigt \textbf{Eingaben von ausserhalb des Lieferlebenszyklus} auf. DAD zeigt, dass vor Beginn des Projekts etwas passiert und dass agile Teams oft neue Anforderungen (in Form von Änderungsanfragen und Fehlermeldungen) aus der Produktion erhalten. Diese Inputs liefern einen wichtigen Kontext für den gesamten Lieferlebenszyklus.

Es gibt eine \textbf{Workitem-Liste} und kein Product Backlog. DAD hat einen grösseren Umfang als Scrum, und wenn man diesen grösseren Umfang berücksichtigt, beginnt man zu erkennen, dass man einen robusteren Change Management-Ansatz benötigt als den Product Backlog von Scrum. Zu den Workitems gehören Anforderungen, Mängel und andere nicht funktionalitätsorientierte Arbeiten wie Schulungen, Ferien und die Unterstützung anderer Teams. Alle diese Arbeiten müssen irgendwie priorisiert werden, nicht nur die Umsetzung der Anforderungen.

Es enthält \textbf{explizite Meilensteine}. Am Ende des Lebenszyklusdiagramms finden sich Hinweise auf vorgeschlagene leichte Meilensteine, die von den Lieferteams angestrebt werden sollten. Solche Meilensteine sind ein wichtiger Aspekt agiler Governance.

DAD findet dann Anwendung, wenn mit verschiedenen Kundengruppen gearbeitet wird und agile Vorgehensmodelle in bestehende Unternehmentrukturen eingeführt werden.

DAD sollte in folgdenden Fällen angewendet werden:

\begin{itemize}
	\item Die Arbeit kann frühzeitig im Projekt identifiziert, priorisiert und geschätzt werden.
	\item Eine gute Wahl für neue agile Teams.
	\item Das Team ist mit Scrum und XP vertraut.
	\item Das Team arbeitet typischerweise an einem Projekt.
\end{itemize}

