\chapter*{Zusammenfassung}
\thispagestyle{fancy}

%TODO 

This lifecycle is based largely upon Scrum and XP with a set of time boxed iterations (sprints) being the core of the Construction phase.  Common scenarios for adopting this version of the lifecycle include situations where you’re extending Scrum to be sufficient for your needs or when you desire to run an “agile project.”

Figure: DAD’s Agile (Scrum based) lifecycle (click to expand).
DAD Agile Project Lifecycle
Features of This Lifecycle

In addition to this being a more detailed view of the lifecycle, there are several interesting aspects to this lifecycle:

    It’s iteration based. Like many agile methods, including both Scrum and XP, the solution is built incrementally in a time-boxed manner. These timeboxes are called iterations (what Scrum calls sprints).
    It uses non-Scrum terminology. Although the lifecycle is Scrum-based we chose to use non-branded terminology in DA, in the case of this diagram the term iteration instead of sprint. The terminology doesn’t really matter, so if you’re more comfortable with Scrum terminology use that instead.
    It shows inputs from outside the delivery lifecycle. Although the overview diagram above showed only the delivery lifecycle, the detailed diagram below shows that something occurs before the project before Inception and that agile teams often get new requirements (in the form of change requests and defect reports) coming in from production. These inputs provide important context for the overall delivery lifecycle.
    There is a work item list, not a product backlog. DA has a greater scope than Scrum, and when you take this greater scope into account you begin to realize you need a more robust change management approach than Scrum’s product backlog. Work items include requirements, defects, and other non-functionality oriented work such as training, vacations, and assisting other teams. All of this work needs to be prioritized somehow, not just implementation of requirements. For more on this, read Agile Best Practice: Prioritized Requirements.
    It includes explicit milestones. Along the bottom of the lifecycle diagram there is an indication of suggested light-weight milestones that delivery teams should strive to meet. Such milestones are an important aspect of agile governance.

When to Apply This Lifecycle

It is the most commonly used lifecycle suitable in these types of situations:

    The work is primarily enhancements or new features
    The work can be identified, prioritized, and estimated early in the project
    A good choice for new agile teams
    The team is familiar with Scrum and XP
    The team is typically working on a project

