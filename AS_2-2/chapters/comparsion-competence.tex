\chapter{Gegenüberstellung Zusammenarbeit}
\thispagestyle{fancy}
\section{Zusammenarbeit}

Das Thema Zuständigkeit wird unter folgenden fünf Aspekten betrachtet:
\begin{enumerate}
\item 	Koordination im Team 
\item 	Rolle und Aufgaben des Kunden \smallskip
		Da in linearen Vorgehensmodellen der Kunde nur an bestimmten Meilensteinen beteiligt ist, interessiert es zu betrachten welche Aufgaben der Kunde hier hat.
\end{enumerate}


\subsection{Koordination im Team}

{\Large Scrum:} \cite{planningReleaseScrum} \medskip

Scrum geht grundsätzlich von einer selbstorganisierenden Koordination aus. Einzig der Scrum Master hat eine "koordinierende" Rolle, jedoch nur im Bezug auf den Prozess und Schnittstellen ausserhalb des Projekts.
Als wichtigstes Instrument für Koordination in Scrum oder auch allgemein in agilen Vorgehensmodellen ist die Kommunikation mit den anderen Team-Mitgliedern. Dies wird in Scrum mit regelmässigen Treffen/Aussprachen realisiert. Wie zum Beispiel das Daily Stand-up oder Scrum of Scrums (Team übergreifend), welche speziell für die Koordination angedacht ist. Weiter Routinen wie Daily Scrum oder Reflektion dienen auch der Koordination und deren Verbesserung, auch wenn das nicht primär ihr Ziel ist.
\bigskip 

{\Large DAD:} \cite{planningReleaseDad} \medskip

DAD möchte, dass folgende Fragen bezüglich Koordination in einem Scrum-Team geklärt sind um eine effektive Organisation innerhalb des Teams zu haben.
\begin{itemize}
	\item 	Wie werden Informationen innerhalb des Teams ausgetauscht?
	\item 	Wer darf die vom Team erstellten Artefakte aktualisieren? 
	\item 	Wie werden wir uns innerhalb des Teams koordinieren?
    \item 	Wenn wir Teil eines größeren Teams sind, wie werden wir dann innerhalb dieses Teams koordinieren?
    \item 	Wie werden wir mit Enterprise-/IT-Teams wie Enterprise Architects und Data Managern zusammenarbeiten?
    \item 	Wie werden wir unsere Release-/Einsatzplanung mit dem Rest des Unternehmens koordinieren?
    \item 	Wenn wir geografisch verteilte Teammitglieder haben, wie werden wir dann mit ihnen zusammenarbeiten?
\end{itemize}
Innerhalb des Teams soll also klar definiert sein, mit welchen Routinen (tägliches Treffen, Video-Konferenzen, usw.) Informationen zwischen den Teammitgliedern ausgetauscht werden und welche Tools dazu verwendet werden. Auch soll auch die Zuständigkeit geregelt sein wer die finalen Artefakten verwaltet, das es hier keine Überschneidungen oder Unklarheiten gibt.
\medskip

Weiter müssen gerade bei agilen Teams innerhalb eines Unternehmen auch die Koordination mit anderen Teams/Abteilungen geregelt sein.
\medskip

Was oft vernachlässigt wird, ist die Koordination mit Teammitglieder an anderen geografischen Orten. Der Informationsaustausch wird hier anspruchsvoller, da direkte verbale Kommunikation, welche die effektivste ist, nicht möglich ist.


\subsection{Rolle und Aufgaben des Kunden}

{\Large Scrum:} \cite{planningPrioScrum} \medskip

Der Kunde soll in Scrum während des ganzen Projektes immer involviert sein. Folgende Möglichkeiten gibt es den Kunden zu involvieren:
\begin{itemize}
	\item Der Kunde wird zum im Initial-Meeting mit einladen.
	\item Backlog wird zusammen mit dem Kunden verwaltet.
	\item Der Kunde nimmt auch an Reviews teil, um Arbeiten zu besprechen und als abgeschlossen zu definieren.
\end{itemize}
Durch den konsequenten Miteinbezug des Kunden wird das Risiko vermindert, dass der Kunde nicht zufrieden ist, da er fortlaufend Einfluss nehmen kann und mit seiner Teilnahme auch frühe Schritte/Arbeiten bestätigt.
\bigskip 

{\Large DAD:} \cite{planningPrioDad} \medskip

DAD hat bezüglich dem Kunden eine andere auch radikalere Haltung. In DAD will man den Begriff "Kunde" nicht verwenden, sondern nur Stakeholder. Dieser werden in folgende Gruppen unterteilt:
\begin{itemize}
	\item Endbenutzer: Personen die das Produkt schlussendlich verwenden.
	\item Vorstehende: Personen die schlussendlich entscheiden, welches Produkt beschafft wird, Bezahlungen freigeben, usw. werden
	\item Partner: Unterhalter, Betreiber, Entwickler von externen Systemen, Juristen, usw.
	\item Interne: Personen innerhalb des Entwicklungsteams und welche technische oder geschäftliche Dienste liefern
\end{itemize}
Für ein Produkt gibt es nur Stakeholders und es gilt dessen Anforderung genau zu ermitteln und festzulegen. Dazu werden für alle Stakeholder die Bedürfnisse gleichwertig ermittelt und miteinbezogen. \smallskip

Man will also bewusst ein Produkt, dass alle Stakeholder gleich berücksichtigen und nicht nur den bezahlenden Kunden" hauptsächlich priorisieren. Denn nur so wird der abbezahlende Kunde" auch ein nachhaltiges und erfolgreiches Produkt erhalten können.
\medskip
Als wichtig wird auch herausgehoben, dass das Projekt im "wir"-Kontext betrachtet wird und nicht im "ihr". Es gibt nicht den Kunden und das Entwicklungsteam, sondern das Projekt betrifft alle gleich.


\section{Empfehlung}

Grundsätzlich wäre der Ansatz mit DAD sicher spannend. Ihr Team kann nach wie vor mit Scrum arbeiten und Sie haben die Möglichkeit mit den erwähnten Instrumenten wie leichte Meilensteine, Release Planung der agilen Entwicklung Einfluss zu nehmen und dem ganzen einen ein Rahmen zu geben. \newline
Jedoch wird Sie das Einführen dieses Vorgehensmodell hohen Aufwand kosten. Vor allem muss für das saubere Anwenden von DAD viel Analyse von bestehenden Prozessen und Zuständen innerhalb Ihres Unternehmens betrieben werden.



	