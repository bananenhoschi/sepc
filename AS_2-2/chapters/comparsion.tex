\chapter{Gegenüberstellung}
\thispagestyle{fancy}
\section{Planung}

Das Thema Planung wird unter folgenden fünf Aspekten betrachtet:
\begin{enumerate}
\item Release-Planung
\item Priorisierung
\item Planungssicherheit
\item Planungsaufwands
\item Nachvollziehbarkeit
\end{enumerate}

\subsection{Release-Planung}

{\Large Scrum:} \cite{planningReleaseScrum} \medskip

Der Release-Plan ist ein höhergestellter Plan, der mehrere Sprint beinhaltet und während der Release-Planung festgelegt wird. Der Plan definiert welche Features umgesetzt werden und wann diese erfüllt sind. Er dient auch dazu, den Fortschritt innerhalb des Projekts verfolgen zu können. Es können mehrere Releases während des Projekt geplant werden, oder einfach ein finales Release am Ende des Projekts. \medskip

Um eine Relaese-Planung durchführen zu können muss folgendes bekannt sein:
\begin{itemize}
\item Ein priorisiertes Scrum-Backlog
\item Die Ressourcen des Scrum-Teams
\item Zielerfüllungsbedingungen
\end{itemize}
Ein Release-Plan kann Termin- oder Feature-geführt sein.\smallskip

Bei Termin-geführten Projekten wird spezifiziert, welche Features bis zu einem bestimmten Termin erfüllt werden können.\smallskip

Bei Feature-geführten Projekten wird spezifiziert , bis zu welchem Termin das Features erfüllt ist.\smallskip

Wie der Backlog ist auch der Release-Plan bei Scrum nicht statisch. Dieser kann sich mit dem Backlog ändern oder auch nach jedem Sprint wieder diskutiert und überarbeitet werden.
\bigskip 

{\Large DAD:} \cite{planningReleaseDad} \medskip

In DAD wird die Release Planung initial in der Inception Phase gemacht. Der Leitfaden empfiehlt für die Releaseplanung folgende sechs Fragen zu beantworten.
\begin{itemize}
	\item Wer wird an der Planung beteiligt sein?
	\item Was ist der Umfang unseres Planungsaufwands?
	\item Was ist unsere Gesamtstrategie, die diesen Plan vorantreibt?
    \item Wie detailliert sollte unser Plan sein?
    \item Welche Kadenzen wird das Team annehmen?
    \item Welchen Ansatz zur Schätzung werden wir wählen?
\end{itemize}
Damit soll sichergestellt werden, dass grundlegende Managementfragen gegenüber den Stakeholder beantwortet sind. Zudem wird erreicht, dass eine durchführbare Strategie besteht und zwischen Stakeholder und Delivery Team ein gemeinsames Verständnis existiert.


\subsection{Priorisierung}

{\Large Scrum:} \cite{planningPrioScrum} \medskip

Das Scrum-Team priorisiert zusammen mit dem Product-Owner die Tasks/Stories aus dem Scurm-Backlog. Wichtig dabei ist, dass nicht nur priorisiert, sondern dass auch sortiert werden muss. Beim Sortieren wird auch die Reihenfolge von Abläufen berücksichtigt. Die Priorisierung geht mit der Sortierung Hand in Hand. \smallskip

Weiter achtet Scrum auch darauf, dass die wertvollsten Inkremente frühstmöglich umgesetzt werden.\bigskip 

{\Large DAD:} \cite{planningPrioDad} \medskip

Die Priorisierung bei DAD verhält sich ähnlich wie die Release-Planung. Grundsätzlich gilt wieder das Rolling-Wave-Modell. Dass heisst, dass höher priorisierte Features detaillierter spezifiziert werden und tief priorisierte nur grob. Die Priorisierung kann zu jederzeit wieder angepasst werden.


\subsection{Planungsumfang}

{\Large Scrum:} \medskip

In Scrum betrifft der Umfang immer direkt das Produkt. Der Scope beinhaltet Features ausgedrückt z.B. als User Stories. Diese sind Im Scrum-Backlog abgelegt und verwaltet.
\bigskip 

{\Large DAD:} \cite{planningScopeDad} \medskip

DAD geht hier einen Schritt weiter und definiert nicht nur Features sondern sogenannte Working-Items. Bei denen werden auch nicht-funktionale Anforderungen definiert wie z.B. Schulungen, Ferien, Unterstützung anderer Teams usw.	


\subsection{Planungsaufwand}

{\Large Scrum:} \medskip

Der Planungsaufwand von Scrum ist relativ gering und ist eigentlich im iterierenden Prozess von Scrum bereits integriert. Die Planung wird bei Scrum vor jedem Sprint im sogenannten Sprint Planning gemacht. Dabei werden die zu erledigenden Items definiert. Der Umfang des Sprints wird vom ganzen Team bestätigt.\bigskip 

{\Large DAD:} \medskip

Initial ist der Planungsaufwand bei DAD hoch. Man muss nebst der eigentlichen Planung des Produkts auch diverse Analysen von Ist-Zuständen bezüglich Ressourcen und Zuständen innerhalb des Unternehmens machen um die Rahmenbedingungen für das Projekt zu legen. Während der Construction Phase ist der Planungsaufwand anlog jenem von Scrum.


\subsection{Risikomanagement}

{\Large Scrum:} \medskip

Bei Scrum wird das Risikomanagement hauptsächlich durch die Kommunikation zwischen dem Kunden und Team geführt. Dies geht über den Product Owner. Dabei muss der Kunde durch seinen stetigen Einfluss mögliche Risiken ausschliessen können. Er kann dies mittels Akzeptanzkriterien beeinflussen.\newline
Aus Team-interner Sicht ist die Definition of Done das Kontrollinstrument um Qualität aber auch Vollständigkeit sicherzustellen.
Als weiteres Instrument dient die Review am Ende jedes Sprints. An jener wird dem Kunde die umgsetzten Items präsentiert. Hier kann der Kunde oder der Product Owner Einfluss bzw. Missverständnisse aufdecken und klären.

\bigskip 

{\Large DAD:} \medskip

Um das Risiko von Fehlkommunikation zu verringern werden bei DAD gegenüber Scrum leichte Meilenstein eingeführt, bei denen ein Abgleich mit dem Kunden stattfindet. \medskip

Weiter sieht DAD die Planung von festen Releases vor. Damit soll regelmässig Software zur Verfügung gestellt werden um ein Feedback des Kunden zu erhalten und frühzeitig festzustellen, ob man die Anforderungen so erfüllen kann. Dies ist gleich wie bei Scrum.

\section{Zuständigket und Zusammenarbeit}
\subsection{Rollen}
{\Large Scrum:} \medskip
{\Large DAD:} \medskip


\section{Empfehlung}

Grundsätzlich wäre der Ansatz mit DAD sicher spannend. Ihr Team kann nach wie vor mit Scrum arbeiten und Sie haben die Möglichkeit mit den erwähnten Instrumenten wie leichte Meilensteine, Release Planung der agilen Entwicklung Einfluss zu nehmen und dem ganzen einen ein Rahmen zu geben. \newline
Jedoch wird Sie das Einführen dieses Vorgehensmodell hohen Aufwand kosten. Vor allem muss für das saubere Anwenden von DAD viel Analyse von bestehenden Prozessen und Zuständen innerhalb Ihres Unternehmens betrieben werden.



	