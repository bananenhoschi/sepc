\chapter{Gegenüberstellung}
\thispagestyle{fancy}
\section{Planung}

Das Thema Planung wird unter folgenden fünf Aspekten betrachtet:
\begin{enumerate}
\item Release-Planung
\item Priorisierung
\item Planungssicherheit
\item Planungsaufwands
\item Nachvollziehbarkeit
\end{enumerate}

\subsection{Release-Planung}

{\Large Scrum:} \cite{planningReleaseScrum} \medskip

Der Release-Plan ist ein höhergestellter Plan, der mehrere Sprints beinhaltet und während der Release-Planung festgelegt wird. Der Plan definiert welche Features umgesetzt werden und wann diese erfüllt sind. Er dient auch dazu, den Fortschritt innerhalb des Projekts verfolgen zu können. Es können mehrere Releases während des Projekts geplant werden, oder einfach ein finales Release am Ende des Projekts. \medskip

Um eine Release-Planung durchführen zu können muss folgendes bekannt sein:
\begin{itemize}
\item Ein priorisiertes Scrum-Backlog
\item Die Ressourcen des Scrum-Teams
\item Zielerfüllungsbedingungen
\end{itemize}
Ein Release-Plan kann Termin- oder Feature-geführt sein.\smallskip

Bei Termin-geführten Projekten wird spezifiziert, welche Features bis zu einem bestimmten Termin erfüllt werden können.\smallskip

Bei Feature-geführten Projekten wird spezifiziert , bis zu welchem Termin das Features erfüllt ist.\smallskip

Wie der Backlog ist auch der Release-Plan bei Scrum nicht statisch. Dieser kann sich mit dem Backlog ändern oder auch nach jedem Sprint wieder diskutiert und überarbeitet werden.
\bigskip 

{\Large DAD:} \cite{planningReleaseDad} \medskip

In DAD wird die Release Planung initial in der Inception Phase gemacht. Der Leitfaden empfiehlt für die Releaseplanung folgende sechs Fragen zu beantworten.
\begin{itemize}
	\item Wer wird an der Planung beteiligt sein?
	\item Was ist der Umfang unseres Planungsaufwands?
	\item Was ist unsere Gesamtstrategie, die diesen Plan vorantreibt?
    \item Wie detailliert sollte unser Plan sein?
    \item Welche Kadenzen wird das Team annehmen?
    \item Welchen Ansatz zur Schätzung werden wir wählen?
\end{itemize}
Damit soll sichergestellt werden, dass grundlegende Managementfragen gegenüber den Stakeholder beantwortet sind. Zudem wird erreicht, dass eine durchführbare Strategie besteht und zwischen Stakeholder und Delivery Team ein gemeinsames Verständnis existiert.


\subsection{Priorisierung}

{\Large Scrum:} \cite{planningPrioScrum} \medskip

Das Scrum-Team priorisiert zusammen mit dem Product-Owner die Tasks/Stories aus dem Scrum-Backlog. Wichtig dabei ist, dass nicht nur priorisiert, sondern dass auch sortiert werden muss. Beim Sortieren wird auch die Reihenfolge von Abläufen berücksichtigt. Die Priorisierung geht mit der Sortierung Hand in Hand. \smallskip

Weiter achtet Scrum auch darauf, dass die wertvollsten Inkremente frühstmöglich umgesetzt werden.\bigskip 

{\Large DAD:} \cite{planningPrioDad} \medskip

Die Priorisierung bei DAD verhält sich ähnlich wie die Release-Planung. Grundsätzlich gilt wieder das Rolling-Wave-Modell. Dass heisst, dass höher priorisierte Features detaillierter spezifiziert werden und tief priorisierte nur grob. Die Priorisierung kann jederzeit wieder angepasst werden.


\subsection{Planungsumfang}

{\Large Scrum:} \medskip

In Scrum betrifft der Umfang immer direkt das Produkt. Der Scope beinhaltet Features ausgedrückt z.B. als User Stories. Diese sind im Scrum-Backlog abgelegt und verwaltet.
\bigskip 

{\Large DAD:} \cite{planningScopeDad} \medskip

DAD geht hier einen Schritt weiter und definiert nicht nur Features sondern sogenannte Working-Items. Bei denen werden auch nicht-funktionale Anforderungen definiert wie z.B. Schulungen, Ferien, Unterstützung anderer Teams usw.	


\subsection{Planungsaufwand}

{\Large Scrum:} \medskip

Der Planungsaufwand von Scrum ist relativ gering und ist eigentlich im iterierenden Prozess von Scrum bereits integriert. Die Planung wird bei Scrum vor jedem Sprint im sogenannten Sprint Planning gemacht. Dabei werden die zu erledigenden Items definiert. Der Umfang des Sprints wird vom ganzen Team bestätigt.\bigskip 

{\Large DAD:} \medskip

Initial ist der Planungsaufwand bei DAD hoch. Man muss nebst der eigentlichen Planung des Produkts auch diverse Analysen von Ist-Zuständen bezüglich Ressourcen und Zuständen innerhalb des Unternehmens machen um die Rahmenbedingungen für das Projekt zu legen. Während der Construction Phase ist der Planungsaufwand analog demjenigen von Scrum.


\subsection{Risikomanagement}

{\Large Scrum:} \medskip

Bei Scrum wird das Risikomanagement hauptsächlich durch die Kommunikation zwischen dem Kunden und Team geführt. Diese geht über den Product Owner. Dabei muss der Kunde durch seinen stetigen Einfluss mögliche Risiken ausschliessen können. Er kann dies mittels Akzeptanzkriterien beeinflussen.\newline
Aus teaminterner Sicht ist die Definition of Done das Kontrollinstrument, um Qualität aber auch Vollständigkeit sicherzustellen.
Als weiteres Instrument dienen die Reviews am Ende jedes Sprints. An jenen werden dem Kunden die umgesetzten Items präsentiert und der Kunde oder der Product Owner kann Einfluss nehmen bzw. Missverständnisse aufdecken und klären.

\bigskip 

{\Large DAD:} \medskip

Um das Risiko von Fehlkommunikation zu verringern werden bei DAD gegenüber Scrum leichte Meilensteine eingeführt, bei denen ein Abgleich mit dem Kunden stattfindet. \medskip

Weiter sieht DAD die Planung von festen Releases vor. Damit soll regelmässig Software zur Verfügung gestellt werden um ein Feedback des Kunden zu erhalten und frühzeitig festzustellen, ob man die Anforderungen so erfüllen kann. Dies ist gleich wie bei Scrum.

\section{Zusammenarbeit}

Das Thema Zuständigkeit wird unter folgenden fünf Aspekten betrachtet:
\begin{enumerate}
	\item 	Koordination im Team 
	\item 	Rolle und Aufgaben des Kunden \smallskip
	Da in linearen Vorgehensmodellen der Kunde nur an bestimmten Meilensteinen beteiligt ist, interessiert es zu betrachten welche Aufgaben der Kunde hier hat.
\end{enumerate}


\subsection{Koordination im Team}

{\Large Scrum:} \cite{planningReleaseScrum} \medskip

Scrum geht grundsätzlich von einer selbstorganisierenden Koordination aus. Einzig der Scrum Master hat eine "koordinierende" Rolle, jedoch nur im Bezug auf den Prozess und Schnittstellen ausserhalb des Projekts.
Als wichtigstes Instrument für Koordination in Scrum oder auch allgemein in agilen Vorgehensmodellen ist die Kommunikation mit den anderen Team-Mitgliedern. Dies wird in Scrum mit regelmässigen Treffen/Aussprachen realisiert. Wie zum Beispiel das Daily Stand-up oder Scrum of Scrums (Team übergreifend), welche speziell für die Koordination angedacht ist. Weiter Routinen wie Daily Scrum oder Reflektion dienen auch der Koordination und deren Verbesserung, auch wenn das nicht primär ihr Ziel ist.
\bigskip 

{\Large DAD:} \cite{planningReleaseDad} \medskip

DAD möchte, dass folgende Fragen bezüglich Koordination in einem Scrum-Team geklärt sind um eine effektive Organisation innerhalb des Teams zu haben.
\begin{itemize}
	\item 	Wie werden Informationen innerhalb des Teams ausgetauscht?
	\item 	Wer darf die vom Team erstellten Artefakte aktualisieren? 
	\item 	Wie werden wir uns innerhalb des Teams koordinieren?
	\item 	Wenn wir Teil eines größeren Teams sind, wie werden wir dann innerhalb dieses Teams koordinieren?
	\item 	Wie werden wir mit Enterprise-/IT-Teams wie Enterprise Architects und Data Managern zusammenarbeiten?
	\item 	Wie werden wir unsere Release-/Einsatzplanung mit dem Rest des Unternehmens koordinieren?
	\item 	Wenn wir geografisch verteilte Teammitglieder haben, wie werden wir dann mit ihnen zusammenarbeiten?
\end{itemize}
Innerhalb des Teams soll also klar definiert sein, mit welchen Routinen (tägliches Treffen, Video-Konferenzen, usw.) Informationen zwischen den Teammitgliedern ausgetauscht werden und welche Tools dazu verwendet werden. Auch soll auch die Zuständigkeit geregelt sein wer die finalen Artefakten verwaltet, das es hier keine Überschneidungen oder Unklarheiten gibt.
\medskip

Weiter müssen gerade bei agilen Teams innerhalb eines Unternehmen auch die Koordination mit anderen Teams/Abteilungen geregelt sein.
\medskip

Was oft vernachlässigt wird, ist die Koordination mit Teammitglieder an anderen geografischen Orten. Der Informationsaustausch wird hier anspruchsvoller, da direkte verbale Kommunikation, welche die effektivste ist, nicht möglich ist.


\subsection{Rolle und Aufgaben des Kunden}

{\Large Scrum:} \cite{planningPrioScrum} \medskip

Der Kunde soll in Scrum während des ganzen Projektes immer involviert sein. Folgende Möglichkeiten gibt es den Kunden zu involvieren:
\begin{itemize}
	\item Der Kunde wird zum im Initial-Meeting mit einladen.
	\item Backlog wird zusammen mit dem Kunden verwaltet.
	\item Der Kunde nimmt auch an Reviews teil, um Arbeiten zu besprechen und als abgeschlossen zu definieren.
\end{itemize}
Durch den konsequenten Miteinbezug des Kunden wird das Risiko vermindert, dass der Kunde nicht zufrieden ist, da er fortlaufend Einfluss nehmen kann und mit seiner Teilnahme auch frühe Schritte/Arbeiten bestätigt.
\bigskip 

{\Large DAD:} \cite{planningPrioDad} \medskip

DAD hat bezüglich dem Kunden eine andere auch radikalere Haltung. In DAD will man den Begriff "Kunde" nicht verwenden, sondern nur Stakeholder. Dieser werden in folgende Gruppen unterteilt:
\begin{itemize}
	\item Endbenutzer: Personen die das Produkt schlussendlich verwenden.
	\item Vorstehende: Personen die schlussendlich entscheiden, welches Produkt beschafft wird, Bezahlungen freigeben, usw. werden
	\item Partner: Unterhalter, Betreiber, Entwickler von externen Systemen, Juristen, usw.
	\item Interne: Personen innerhalb des Entwicklungsteams und welche technische oder geschäftliche Dienste liefern
\end{itemize}
Für ein Produkt gibt es nur Stakeholders und es gilt dessen Anforderung genau zu ermitteln und festzulegen. Dazu werden für alle Stakeholder die Bedürfnisse gleichwertig ermittelt und miteinbezogen. \smallskip

Man will also bewusst ein Produkt, dass alle Stakeholder gleich berücksichtigen und nicht nur den bezahlenden Kunden" hauptsächlich priorisieren. Denn nur so wird der abbezahlende Kunde" auch ein nachhaltiges und erfolgreiches Produkt erhalten können.
\medskip
Als wichtig wird auch herausgehoben, dass das Projekt im "wir"-Kontext betrachtet wird und nicht im "ihr". Es gibt nicht den Kunden und das Entwicklungsteam, sondern das Projekt betrifft alle gleich.


\section{Empfehlung}

Grundsätzlich wäre der Ansatz mit DAD sicher spannend. Ihr Team kann nach wie vor mit Scrum arbeiten und Sie haben die Möglichkeit mit den erwähnten Instrumenten wie leichte Meilensteine, Release Planung der agilen Entwicklung Einfluss zu nehmen und dem ganzen einen ein Rahmen zu geben. \newline
Jedoch wird Sie das Einführen dieses Vorgehensmodell hohen Aufwand kosten. Vor allem muss für das saubere Anwenden von DAD viel Analyse von bestehenden Prozessen und Zuständen innerhalb Ihres Unternehmens betrieben werden.



	