\chapter{Gegenüberstellung}
\thispagestyle{fancy}
\section{Planung}

Das Thema Planung wird unter folgenden fünf Aspekten betrachtet:
\begin{enumerate}
\item Release-Planung
\item Priorisierung
\item Planungssicherheit
\item Planungsaufwands
\item Nachvollziehbarkeit
\end{enumerate}

\subsection{Release-Planung}

{\Large Scrum:} \cite{planningReleaseScrum} \medskip

Der Release-Plan ist ein höhergestellter Plan, der mehrere Sprint beinhaltet und während der Release-Planung festgelegt wird. Der Plan definiert welche Features umgesetzt werden und wann diese erfüllt sind. Er dient auch dazu, den Fortschritt innerhalb des Projekts verfolgen zu können. Es können mehrere Releases während des Projekt geplant werden, oder einfach ein finales Release am Ende des Projekts. \medskip

Um eine Relaese-Planung durchführen zu können muss folgendes bekannt sein:
\begin{itemize}
\item Ein priorisiertes Scrum-Backlog
\item Die Ressourcen des Scrum-Teams
\item Zielerfüllungsbedingungen
\end{itemize}
Ein Release-Plan kann Termin- oder Feature-geführt sein.\smallskip

Bei Termin-geführten Projekten wird spezifiziert, welche Features bis zu einem bestimmten Termin erfüllt werden können.\smallskip

Bei Feature-geführten Projekten wird spezifiziert , bis zu welchem Termin das Features erfüllt ist.\smallskip

Wie der Backlog ist auch der Release-Plan bei Scrum nicht statisch. Dieser kann sich mit dem Backlog ändern oder auch nach jedem Sprint wieder diskutiert und überarbeitet werden.
\bigskip 

{\Large DAD:} \cite{planningReleaseDad} \medskip

Grundsätzlich kann im DAD wieder die gleiche Release Planung wie in Scrum angewendet werdet. DAD empfiehlt aber für erfahrene DAD-Anwender ein Rolling-Wave-Modell für die Release-Planung. \smallskip

Ein Rolling-Wave-Modell ist, dass Features die man Zeitnah umsetzten will, genauer ausarbeitet und spezifiziert. Features in weiter entfernter Zeit nur grob spezifiziert.  Grund dazu ist, das Features in fernere Zukunft sich eher ändern können und somit  Zeit zu früh in die Ausarbeitung verschwendet sein könnte. Man möchte die ferneren Features jedoch trotzdem grob spezifizieren, um jetzige Entscheidungen zu begründen und auch die Erwartungen seitens Kunde zu erfüllen.


\subsection{Priorisierung}

{\Large Scrum:} \cite{planningPrioScrum} \medskip

Das Scrum-Team priorisiert zusammen mit dem Product-Owner die Tasks/Stories aus dem Scurm-Backlog. Wichtig dabei ist, dass nicht nur priorisiert , sondern dass auch sortiert werden muss. Beim Sortieren wird auch die Reihenfolge von Abläufen berücksichtigt. Priorisierung geht mit der Sortierung Hand in Hand. \smallskip

Weiter setzt Scrum auch den Ansatz, dass die wertvollsten 
\bigskip 

{\Large DAD:} \cite{planningPrioDad} \medskip

Die Priorisierung bei DAD verhält sich ähnlich wie die Release-Planung. Grundsätzlich gilt wieder das Rolling-Wave-Modell. Dass heisst, dass höher priorisierte Features detaillierter spezifiziert werden und tief priorisierte nur grob. Die Priorisierung kann zu jederzeit wieder Angepasst werden.


\subsection{Planungsumfang}

{\Large Scrum:} \medskip

In Scrum betrifft der Umfang immer direkt das Produkt. Der Scope beinhaltet Features ausgedrückt z.B. als User Stories. Diese sind Im Scrum-Backlog abgelegt und verwaltet. \smallskip

Weiter setzt Scrum auch den Ansatz, dass die wertvollsten 
\bigskip 

{\Large DAD:} \cite{planningScopeDad} \medskip

DAD geht hier einen Schritt weiter und definiert nicht nur Features sondern sogenannte Working-Items. Bei denen werden auch nicht-funktionale Anforderungen definiert wie z.B. Schulungen, Ferien, Unterstützung anderer Teams usw.	


\subsection{Planungsaufwand}

{\Large Scrum:} \medskip

Der Planungsaufwand von Scrum ist relativ gering und ist eigentlich im iterierenden Prozess von Scrum bereits integriert und wird immer wieder angewandt.
\bigskip 

{\Large DAD:} \medskip

Wird DAD voll umfänglich eingesetzt ist der initiale Planungsaufwand hoch. Man muss nebst der eigentlichen Planung des Produkts auch diverse Analysen von Ist-Zuständen bezüglich Ressourcen und Zuständen innerhalb des Unternehmens machen um die Rahmenbedingungen für das Projekt zu legen.


\subsection{Risikomanagement}

{\Large Scrum:} \medskip

Bei Scrum wird das Risikomanagement hauptsächlich durch die Kommunikation zwischen dem Kunden und Team geführt. Dabei muss der Kunde durch seinen stetigen Einfluss mögliche Risiken ausschliessen können. \newline
Werkzeuge dazu sind z.B. die Definition of Done, also wann akzeptiere ich etwas als abgeschlossen.

\bigskip 

{\Large DAD:} \medskip

Um das Risiko von Fehlkommunikation zu verringern werden bei DAD gegenüber Scrum leichte Meilenstein eingeführt, bei denen einen Abgleich mit dem Kunden stattfindet. \medskip

Weiter sieht DAD die Planung von festen Releases vor. Damit soll regelmässig Software zur Verfügung gestellt werden um ein Feedback des Kunden zu erhalten und frühzeitig festzustellen, ob man die Anforderungen so erfüllen kann.



\subsection{Empfehlung}

Grundsätzlich wäre der Ansatz mit DAD sicher spannend. Ihr Team kann nach wie vor mit Scrum arbeiten und Sie haben die Möglichkeit mit den erwähnten Instrumenten wie leichte Meilensteine, Release Planung der agilen Entwicklung Einfluss zu nehmen und dem ganzen einen ein Rahmen zu geben. \newline
Jedoch wird Sie das Einführen dieses Vorgehensmodell hohen Aufwand kosten. Vor allem muss für das saubere Anwenden von DAD viel Analyse von bestehenden Prozessen und Zuständen innerhalb Ihres Unternehmens betrieben werden.



	