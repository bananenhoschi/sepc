\chapter{Empfehlung}


Grundsätzlich wäre der Ansatz mit DAD sicher spannend. Ihr Team kann nach wie vor mit Scrum arbeiten und Sie haben die Möglichkeit mit den erwähnten Instrumenten wie leichten Meilensteinen oder der Release Planung Einfluss auf die agile Entwicklung zu nehmen. Somit können Sie der Entwicklung den nötigen Freiraum geben, aber trotzdem eine gezielte Steuerung vornehmen.

Jedoch wird das Einführen dieses Vorgehensmodell vergleichsweise hohe Aufwände verursachen. Vor allem muss für das saubere Anwenden von DAD viel Analyse von bestehenden Prozessen und Zuständen innerhalb Ihres Unternehmens betrieben werden.

Die Gegenüberstellung von Scrum und DAD hat gezeigt, dass mit beiden Modellen Vor- aber auch Nachteile entstehen. Aus unserer Sicht überwiegen für Ihre Unternehmung die Vorteile von DAD. Diese sollen im Folgenden nochmals Zusammenfassend aufgezeigt werden:

Im Bereich der Releaseplanung bietet DAD den Vorteil, dass die Release Planung bereits frühzeitig im Projekt definiert wird. Gegenüber Scrum werden hier viel tiefergehende Fragen geklärt um das Projekt im Einklang mit der unternehmerischen Strategie zu bringen. Dabei wird früh mit den beteiligten Stakeholdern ein gemeinsames Verständnis gefunden, nach welchem das weitere Vorgehen definiert ist. Scrum konzentriert sich hier mehr auf das Projekt und klammert unternehmerische Fragen grösstenteils aus.

Die Priorisierung von Scrum und DAD verhält sich sehr ähnlich. Bei beiden wird iterativ eine Priorisierung der Inkremente vorgenommen, wobei die wichtigsten Features jeweils den Vortritt erhalten. DaD geht zusätzlich nach dem Rolling-Wave-Modell vor und spezifiziert höher priorisierte Features detaillierter als tiefer priorisierte. Wird dieser Vorgang angewendet, so geschieht im Projekt Team dadurch eine Fokussierung aufs Wesentliche. Die wichtigsten Requirements werden genauer besprochen, während die Details von weniger wichtigen Anforderungen zeitsparend weggelassen werden können.

Der Planungsausfwand bei DAD ist initial sicher höher, dafür bietet der Planungsumfang grosse Vorteile. So werden z.B. auch nicht funktionale Anforderungen definiert, welche über das eigentliche Projekt hinaus gehen. Womit DAD im Vergleich zu Scrum besser mit bestehenden Unternehmensstrukturen und spezifischen Projektanforderungen der Unternehmen vereinbar ist.

Scrum definiert für die Zusammenarbeit verschiedene Rollen innerhalb des Projekt Teams. Das Modell klärt aber keine oder nur wenige Fragen bezüglicher der Koordination mit der Unternehmung und den darin bereits vorhandenen Rollen. Hier bietet DAD einen grossen Vorteil, da es genau auf diese Fragen eingeht. Somit können mit DAD weiterführende Fragen zur Integration des Projekt Teams in die Unternehmung geklärt werden und das Modell optimal im bestehenden Unternehmen integriert werden. So werden im DAD Modell auch einzelne Rollen, die im Scrum in den Aufgabenbereich des Teams fallen, explizit aufgeführt. Diese sind spezifisch auf die Bedürfnisse von bestehenden Firmenstrukturen ausgelegt. Insbesondere ist DAD auf hierarchische Unternehmensstrukturen ausgelegt welche durch die definierten Rollen im Modell übernommen werden können.


Gegenüber dem Kunden gibt es grosse Unterschiede zwischen Scrum und DAD. Scrum priorisiert den zahlenden Kunden, während DAD versucht alle Stakeholder gleichwertig mit einzubeziehen. Daraus ergibt sich mit DAD umfassendere Rückmeldungen zu den einzelnen Releases. Das führt zwar zu einem komplexeren Gesamtbild, dafür können Stakeholder Bedürfnisse früh herauskristallisiert und entsprechend früh in den Entwicklungsprozess miteinbezogen werden. Wird dies dem Kunden gegenüber entsprechend kommuniziert, können dadurch grosse Vorteile, sowie Kosteneinsparungen im Projekt erreicht werden. 

Zusammenfassend kann festgehalten werden, dass Scrum und DAD sich zwar sehr ähnlich sind, DAD aber einige wichtige Vorteile für Ihr Unternehmen bietet. Neben den vielfältigeren Rollen, welche so ausgelegt sind, dass sie in die bestehenden Unternehmensstrukturen eingebettet werden können, werden mit DAD auch Unternehmensstrategische Ziele berücksichtigt. Dies sind wichtige Fragen, die für die beiden bestehenden Unternehmen geklärt werden müssen und genau dafür ist DAD ausgelegt. Somit kann ein Team nach wie vor mit Scrum arbeiten, da DAD eine Erweiterung von Scrum darstellt, und Sie haben die Möglichkeit mit den erwähnten Instrumenten wie leichten Meilensteinen und Releasen Planung gezielt Einfluss auf die Entwicklung zu nehmen.

Ein weiterer Vorteil für die Einführung von DAD ist die Tatsache, dass sich beide Unternehmen auf neue Strukturen einlassen müssen. Was im gegebenen Fall sicher von Vorteil ist. Zwar ist der Planungs- und Transitionsaufwand höher als bei der Übernahme eines bereits verwendeten Modells, doch die gemeinsame Bewältigung der Aufgabe wird den Zusammenhalt der Teams stärken und ein gemeinsames Verständnis zwischen den beiden Unternehmen schaffen. Aus den genannten Gründen empfehlen wir Ihnen unter Berücksichtigung Ihrer bestehenden Strukturen das Modell DAD einzuführen.

